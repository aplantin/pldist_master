\nonstopmode{}
\documentclass[a4paper]{book}
\usepackage[times,inconsolata,hyper]{Rd}
\usepackage{makeidx}
\usepackage[utf8]{inputenc} % @SET ENCODING@
% \usepackage{graphicx} % @USE GRAPHICX@
\makeindex{}
\begin{document}
\chapter*{}
\begin{center}
{\textbf{\huge Package `pldist'}}
\par\bigskip{\large \today}
\end{center}
\begin{description}
\raggedright{}
\inputencoding{utf8}
\item[Title]\AsIs{Paired and Longitudinal Ecological Dissimilarities}
\item[Version]\AsIs{0.0.0.9000}
\item[Author]\AsIs{Anna M. Plantinga [aut, cre], Jun Chen [aut]}
\item[Maintainer]\AsIs{Anna M. Plantinga }\email{amp9@williams.edu}\AsIs{}
\item[Description]\AsIs{Calculates paired and longitudinal UniFrac, Bray-Curtis, Jaccard, Gower, and
Kulczynski distances/dissimilarities. These metrics summarize changes in the microbiome
over time and allow these changes to be compared across treatments, conditions, or
levels of a covariate. For more information, please see Plantinga et al (2018+).}
\item[Depends]\AsIs{R (>= 3.4.3)}
\item[Imports]\AsIs{ape}
\item[License]\AsIs{GPL-3}
\item[Encoding]\AsIs{UTF-8}
\item[LazyData]\AsIs{true}
\item[RoxygenNote]\AsIs{6.1.0}
\item[NeedsCompilation]\AsIs{no}
\item[Suggests]\AsIs{knitr, rmarkdown, testthat, devtools}
\item[VignetteBuilder]\AsIs{knitr}
\item[RemoteType]\AsIs{github}
\item[RemoteHost]\AsIs{https://api.github.com}
\item[RemoteRepo]\AsIs{pldist}
\item[RemoteUsername]\AsIs{aplantin}
\item[RemoteRef]\AsIs{master}
\item[RemoteSha]\AsIs{4e9c178e4a26850877d4529ff96655474602ecde}
\item[GithubRepo]\AsIs{pldist}
\item[GithubUsername]\AsIs{aplantin}
\item[GithubRef]\AsIs{master}
\item[GithubSHA1]\AsIs{4e9c178e4a26850877d4529ff96655474602ecde}
\end{description}
\Rdcontents{\R{} topics documented:}
\inputencoding{utf8}
\HeaderA{bal.long.meta}{Simulated metadata for balanced longitudinal study design.}{bal.long.meta}
\keyword{datasets}{bal.long.meta}
%
\begin{Description}\relax
Simulation code is included in the package vignette. 
Corresponding OTU matrix is stored in `bal.long.otus`.
\end{Description}
%
\begin{Usage}
\begin{verbatim}
data(bal.long.meta)
\end{verbatim}
\end{Usage}
%
\begin{Format}
A data frame with 15 rows and 3 columns. 
\begin{description}

\item[subjID] Subject identifiers
\item[sampID] Sample identifiers, matches row names of OTU count matrix
\item[time] Time indicator

\end{description}
\end{Format}
\inputencoding{utf8}
\HeaderA{bal.long.otus}{Simulated OTU data for balanced longitudinal study design.}{bal.long.otus}
\keyword{datasets}{bal.long.otus}
%
\begin{Description}\relax
Simulation code is included in the package vignette. 
Corresponding metadata is stored in `bal.long.meta`.
\end{Description}
%
\begin{Usage}
\begin{verbatim}
data(bal.long.otus)
\end{verbatim}
\end{Usage}
%
\begin{Format}
A matrix with 15 rows and 10 columns. Rows are samples, columns are OTUs.
\end{Format}
\inputencoding{utf8}
\HeaderA{braycurtis}{Paired or longitudinal Bray-Curtis distances}{braycurtis}
%
\begin{Description}\relax
The distances are calculated as follows, where d\_k\textasciicircum{}X is the within-subject 
measure of change appropriate to the setting (paired/longitudinal and 
quantitative/qualitative), as described in the full package documentation 
and vignette. 
\eqn{D_{AB} = (1/m) * \sum_k |d_k^A - d_k^B|}{}
\end{Description}
%
\begin{Usage}
\begin{verbatim}
braycurtis(tsf.data, binary)
\end{verbatim}
\end{Usage}
%
\begin{Arguments}
\begin{ldescription}
\item[\code{tsf.data}] Transformed OTU table and metadata (from function pl.transform)

\item[\code{binary}] Logical indicating whether to use the binary version of the distance
\end{ldescription}
\end{Arguments}
%
\begin{Value}
Returns an n x n distance matrix.
\end{Value}
\inputencoding{utf8}
\HeaderA{counts2props}{counts2props}{counts2props}
%
\begin{Description}\relax
Converts OTU counts to OTU proportions/relative abundances.
\end{Description}
%
\begin{Usage}
\begin{verbatim}
counts2props(x)
\end{verbatim}
\end{Usage}
%
\begin{Arguments}
\begin{ldescription}
\item[\code{x}] Matrix of OTU counts (rows are subjects, columns are taxa).
\end{ldescription}
\end{Arguments}
%
\begin{Value}
n x p matrix of OTU proportions.
\end{Value}
\inputencoding{utf8}
\HeaderA{flexsign}{flexsign}{flexsign}
%
\begin{Description}\relax
Sign function that considers 0 both positive and negative. Returns 1 if the two numbers are the same sign, 0 otherwise. Vectorized (compares vectors elementwise).
\end{Description}
%
\begin{Usage}
\begin{verbatim}
flexsign(v1, v2)
\end{verbatim}
\end{Usage}
%
\begin{Arguments}
\begin{ldescription}
\item[\code{v1}] First vector

\item[\code{v2}] Second vector
\end{ldescription}
\end{Arguments}
%
\begin{Value}
Returns an n x n distance matrix.
\end{Value}
\inputencoding{utf8}
\HeaderA{gower}{Paired or longitudinal Gower distances}{gower}
%
\begin{Description}\relax
The distances are calculated as follows, where d\_k\textasciicircum{}X is the within-subject 
measure of change appropriate to the setting (paired/longitudinal and 
quantitative/qualitative), as described in the full package documentation 
and vignette. 

\eqn{D_{AB} = (1/m) \sum_k (|d_k^A - d_k^B|)/(\max d_k - \min d_k)}{}
\end{Description}
%
\begin{Usage}
\begin{verbatim}
gower(tsf.data, binary)
\end{verbatim}
\end{Usage}
%
\begin{Arguments}
\begin{ldescription}
\item[\code{tsf.data}] Transformed OTU table and metadata (from function pl.transform)

\item[\code{binary}] Logical indicating whether to use the binary version of the distance
\end{ldescription}
\end{Arguments}
%
\begin{Value}
Returns an n x n distance matrix.
\end{Value}
\inputencoding{utf8}
\HeaderA{jaccard}{Paired or longitudinal Jaccard distances}{jaccard}
%
\begin{Description}\relax
The distances are calculated as follows, where d\_k\textasciicircum{}X is the within-subject 
measure of change appropriate to the setting (paired/longitudinal and 
quantitative/qualitative), as described in the full package documentation 
and vignette. 
Paired, qualitative: \eqn{D_{AB} = 1 - {\sum_k I(d_k^A = d_k^B) I(d_k^A \neq 0)}/{\sum_k [I(d_k^A \neq 0) + I(d_k^B \neq 0)]}}{}
Paired, quantitative: \eqn{D_{AB} = 1 - {\sum_k \min(|d_k^A|, |d_k^B|) \, I(sgn(d_k^A) = sgn(d_k^B))}/{\sum_k \max(|d_k^A|, |d_k^b|)}}{}
Longitudinal: \eqn{D_{AB} = 1 - (\sum_k \min(d_k^A, d_k^B))/(\sum_k \max(d_k^A, d_k^B))}{}
\end{Description}
%
\begin{Usage}
\begin{verbatim}
jaccard(tsf.data, paired, binary)
\end{verbatim}
\end{Usage}
%
\begin{Arguments}
\begin{ldescription}
\item[\code{tsf.data}] Transformed OTU table and metadata (from function pl.transform)

\item[\code{paired}] Logical indicating whether paired analysis is desired

\item[\code{binary}] Logical indicating whether to use the binary version of the distance
\end{ldescription}
\end{Arguments}
%
\begin{Value}
Returns an n x n distance matrix.
\end{Value}
\inputencoding{utf8}
\HeaderA{kulczynski}{Paired or longitudinal Kulczynski distances}{kulczynski}
%
\begin{Description}\relax
The distances are calculated as follows, where d\_k\textasciicircum{}X is the within-subject 
measure of change appropriate to the setting (paired/longitudinal and 
quantitative/qualitative), as described in the full package documentation 
and vignette. 

Paired, qualitative: \eqn{D_{AB} = 1 - (1/m) \sum_k I[d_k^A = d_k^B] I[d_k^A \neq 0]}{} 
Paired, quantitative: \eqn{D_{AB} = 1 - (2/m) \sum_k \min(|d_k^A|, |d_k^B|) I[sgn(d_k^A) = sgn(d_k^B)]}{}
Longitudinal: \eqn{D_{AB} = 1 - (1/m) * \sum_k \min(d_k^A, d_k^B)}{}
\end{Description}
%
\begin{Usage}
\begin{verbatim}
kulczynski(tsf.data, paired, binary)
\end{verbatim}
\end{Usage}
%
\begin{Arguments}
\begin{ldescription}
\item[\code{tsf.data}] Transformed OTU table and metadata (from function pl.transform)

\item[\code{paired}] Logical indicating whether paired analysis is desired

\item[\code{binary}] Logical indicating whether to use the binary version of the distance
\end{ldescription}
\end{Arguments}
%
\begin{Value}
Returns an n x n distance matrix.
\end{Value}
\inputencoding{utf8}
\HeaderA{LUniFrac}{LUniFrac}{LUniFrac}
%
\begin{Description}\relax
Longitudinal UniFrac distances for comparing changes in
microbial communities across 2 time points.
\end{Description}
%
\begin{Usage}
\begin{verbatim}
LUniFrac(otu.tab, tree, gam = c(0, 0.5, 1), metadata)
\end{verbatim}
\end{Usage}
%
\begin{Arguments}
\begin{ldescription}
\item[\code{otu.tab}] OTU count table, containing 2*n rows (samples) and q columns (OTUs)

\item[\code{tree}] Rooted phylogenetic tree of R class "phylo"

\item[\code{gam}] Parameter controlling weight on abundant lineages. The same weight is used within a subjects as between subjects.

\item[\code{metadata}] Data frame with three columns: subject identifiers (n unique values), 
sample identifiers (must match row names of otu.tab), 
and time or group indicator (numeric variable, or factor with levels such that as.numeric returns 
the desired ordering). Column names should be subjID, sampID, time.
\end{ldescription}
\end{Arguments}
%
\begin{Details}\relax
Based in part on Jun Chen \& Hongzhe Li (2012), GUniFrac.

Computes difference between time points and then calculates
difference of these differences, resulting in a dissimilarity
matrix that can be used in a variety of downstream 
distance-based analyses.
\end{Details}
%
\begin{Value}
Returns a (K+1) dimensional array containing the longitudinal UniFrac dissimilarities 
with the K specified gamma values plus the unweighted distance. The unweighted dissimilarity 
matrix may be accessed by result[,,"d\_UW"], and the generalized dissimilarities by result[,,"d\_G"] 
where G is the particular choice of gamma.
\end{Value}
\inputencoding{utf8}
\HeaderA{paired.meta}{Simulated metadata for paired study design.}{paired.meta}
\keyword{datasets}{paired.meta}
%
\begin{Description}\relax
Simulation code is included in the package vignette. 
Corresponding OTU matrix is stored in `paired.otus`.
\end{Description}
%
\begin{Usage}
\begin{verbatim}
data(paired.meta)
\end{verbatim}
\end{Usage}
%
\begin{Format}
A data frame with 10 rows and 3 columns. 
\begin{description}

\item[subjID] Subject identifiers
\item[sampID] Sample identifiers, matches row names of OTU count matrix
\item[time] Time indicator, takes values 1 or 2

\end{description}
\end{Format}
\inputencoding{utf8}
\HeaderA{paired.otus}{Simulated OTU data for paired study design.}{paired.otus}
\keyword{datasets}{paired.otus}
%
\begin{Description}\relax
Simulation code is included in the package vignette. 
Corresponding metadata is stored in `paired.meta`.
\end{Description}
%
\begin{Usage}
\begin{verbatim}
data(paired.otus)
\end{verbatim}
\end{Usage}
%
\begin{Format}
A matrix with 10 rows and 10 columns. Rows are samples, columns are OTUs.
\end{Format}
\inputencoding{utf8}
\HeaderA{pl.transform}{tsf\_paired}{pl.transform}
%
\begin{Description}\relax
OTU transformation for longitudinal data. Computes average within-subject change 
(in presence for qualitative metrics, abundance for quantitative metrics) 
during one unit of time for each taxon.
\end{Description}
%
\begin{Usage}
\begin{verbatim}
pl.transform(otus, metadata, paired)
\end{verbatim}
\end{Usage}
%
\begin{Arguments}
\begin{ldescription}
\item[\code{otus}] Matrix of OTU counts or proportions. Notes: (1) Will be transformed to 
proportions if it's not already; (2) Row names must be sample identifiers 
(matching metadata), and column names must be OTU identifiers (enforced if 
using UniFrac distances).

\item[\code{metadata}] Data frame with three columns: subject identifiers (n unique values, column name "subjID"), 
sample identifiers (must match row names of otu.tab, column name "sampID"), 
and time point or group identifier (if using longitudinal distances, this must be numeric or 
convertable to numeric).

\item[\code{paired}] Logical indicating whether to use the paired version of the metric (TRUE) or the 
longitudinal version (FALSE). Paired analyis is only possible when there are exactly 2 
unique time points/identifiers for each subject or pair.
\end{ldescription}
\end{Arguments}
%
\begin{Value}
List with the following elements. Both data matrices have subject identifiers 
as row names and OTU identifiers as column names.  
\begin{ldescription}
\item[\code{tsf.data}] List with 3 elements: 
(1) dat.binary: n x p matrix of data after longitudinal, binary/qualitative transformation 
(2) dat.quant: n x p matrix of data after longitudinal, quantitative transformation
(3) avg.prop: n x p matrix with overall average proportion of each taxon 
\item[\code{type}] Type of transformation that was used (paired, balanced longitudinal, 
unbalanced longitudinal) with a warning if unbalanced longitudinal.
\end{ldescription}
\end{Value}
\inputencoding{utf8}
\HeaderA{pldist}{pldist}{pldist}
%
\begin{Description}\relax
Function that calculates paired and longitudinal ecological distance/dissimilarity 
matrices. Includes qualitative and quantitative versions of Bray-Curtis, Jaccard, Kulczynski, 
Gower, and unweighted and generalized UniFrac distances/dissimilarities. UniFrac-based 
metrics are based in part on GUniFrac (Jun Chen \& Hongzhe Li (2012)).
\end{Description}
%
\begin{Usage}
\begin{verbatim}
pldist(otus, metadata, paired = FALSE, binary = FALSE, method,
  tree = NULL, gam = c(0, 0.5, 1))
\end{verbatim}
\end{Usage}
%
\begin{Arguments}
\begin{ldescription}
\item[\code{otus}] OTU count or frequency table, containing one row per sample and one column per OTU.

\item[\code{metadata}] Data frame with three columns: subject identifiers (n unique values, column name "subjID"), 
sample identifiers (must match row names of otu.tab, column name "sampID"), 
and time point or group identifier (if using longitudinal distances, this must be numeric or 
convertable to numeric).

\item[\code{paired}] Logical indicating whether to use the paired version of the metric (TRUE) or the 
longitudinal version (FALSE). Paired analyis is only possible when there are exactly 2 
unique time points/identifiers for each subject or pair.

\item[\code{binary}] Logical indicating whether to use the qualitative (TRUE) or quantitative (FALSE) 
version of each metric. Qualitative analysis only incorporates changes in OTU presence or 
absence; quantitative analysis incorporates changes in abundance.

\item[\code{method}] Desired distance metric. Choices are braycurtis, jaccard, kulczynski, gower, and 
unifrac, or any unambiguous abbreviation thereof.

\item[\code{tree}] Rooted phylogenetic tree of R class "phylo". Default NULL; only needed for 
UniFrac family distances.

\item[\code{gam}] Parameter controlling weight on abundant lineages for UniFrac family distances. The 
same weight is used within a subject as between subjects. Default (0, 0.5, 1).
\end{ldescription}
\end{Arguments}
%
\begin{Value}
Returns a list with elements: 
\begin{ldescription}
\item[\code{D}] If any metric other than UniFrac is used, D is an n x n distance (or dissimilarity) matrix. 
For UniFrac-family dissimilarities, D is a (K+1) dimensional array containing the paired or 
longitudinal UniFrac dissimilarities with the K specified gamma values plus the unweighted 
distance. The unweighted distance matrix may be accessed by result[,,"d\_UW"], and the 
generalized dissimilarities by result[,,"d\_G"] where G is the particular choice of gamma.
\item[\code{type}] String indicating what type of dissimilarity was requested.
\end{ldescription}
\end{Value}
\inputencoding{utf8}
\HeaderA{pltransform}{tsf\_paired}{pltransform}
%
\begin{Description}\relax
OTU transformation for longitudinal data. Computes average within-subject change 
(in presence for qualitative metrics, abundance for quantitative metrics) 
during one unit of time for each taxon.
\end{Description}
%
\begin{Usage}
\begin{verbatim}
pltransform(otus, metadata, paired, check.input = TRUE)
\end{verbatim}
\end{Usage}
%
\begin{Arguments}
\begin{ldescription}
\item[\code{otus}] Matrix of OTU counts or proportions. Notes: (1) Will be transformed to 
proportions if it's not already; (2) Row names must be sample identifiers 
(matching metadata), and column names must be OTU identifiers (enforced if 
using UniFrac distances).

\item[\code{metadata}] Data frame with three columns: subject identifiers (n unique values, column name "subjID"), 
sample identifiers (must match row names of otu.tab, column name "sampID"), 
and time point or group identifier (if using longitudinal distances, this must be numeric or 
convertable to numeric).

\item[\code{paired}] Logical indicating whether to use the paired version of the metric (TRUE) or the 
longitudinal version (FALSE). Paired analyis is only possible when there are exactly 2 
unique time points/identifiers for each subject or pair.

\item[\code{check.input}] Logical indicating whether to check input values (default TRUE).
\end{ldescription}
\end{Arguments}
%
\begin{Value}
List with the following elements. Both data matrices have subject identifiers 
as row names and OTU identifiers as column names.  
\begin{ldescription}
\item[\code{tsf.data}] List with 3 elements: 
(1) dat.binary: n x p matrix of data after longitudinal, binary/qualitative transformation 
(2) dat.quant: n x p matrix of data after longitudinal, quantitative transformation
(3) avg.prop: n x p matrix with overall average proportion of each taxon 
\item[\code{type}] Type of transformation that was used (paired, balanced longitudinal, 
unbalanced longitudinal) with a warning if unbalanced longitudinal.
\end{ldescription}
\end{Value}
\inputencoding{utf8}
\HeaderA{PUniFrac}{PUniFrac}{PUniFrac}
%
\begin{Description}\relax
Paired UniFrac distances for comparing changes in
microbial communities across 2 groups or time points.
\end{Description}
%
\begin{Usage}
\begin{verbatim}
PUniFrac(otu.tab, tree, gam = c(0, 0.5, 1), metadata)
\end{verbatim}
\end{Usage}
%
\begin{Arguments}
\begin{ldescription}
\item[\code{otu.tab}] OTU count table, containing 2*n rows (samples) and q columns (OTUs)

\item[\code{tree}] Rooted phylogenetic tree of R class "phylo"

\item[\code{gam}] Parameter controlling weight on abundant lineages. The same weight is used within a subject as between subjects.

\item[\code{metadata}] Data frame with three columns: subject identifiers (n unique values, column name "subjID"), 
sample identifiers (must match row names of otu.tab, column name "sampID"), 
and time point or group identifier (variable with two unique levels, column name "time").
\end{ldescription}
\end{Arguments}
%
\begin{Details}\relax
Based in part on Jun Chen \& Hongzhe Li (2012), GUniFrac.

Computes difference between time points and then calculates
difference of these differences, resulting in a dissimilarity
matrix that can be used in a variety of downstream 
distance-based analyses.
\end{Details}
%
\begin{Value}
Returns a (K+1) dimensional array containing the longitudinal UniFrac dissimilarities 
with the K specified gamma values plus the unweighted distance. The unweighted dissimilarity 
matrix may be accessed by result[,,"d\_UW"], and the generalized dissimilarities by result[,,"d\_G"] 
where G is the particular choice of gamma.
\end{Value}
\inputencoding{utf8}
\HeaderA{sim.tree}{Simulated rooted phylogenetic tree.}{sim.tree}
\keyword{datasets}{sim.tree}
%
\begin{Description}\relax
Simulation code is included in the package vignette. 
Tree includes 10 OTUs and may be used with any of the 
simulated data sets (paired, balanced longitudinal, or 
unbalanced longitudinal).
\end{Description}
%
\begin{Usage}
\begin{verbatim}
data(sim.tree)
\end{verbatim}
\end{Usage}
%
\begin{Format}
An object of class "phylo".
\end{Format}
\inputencoding{utf8}
\HeaderA{tsf\_long}{tsf\_paired}{tsf.Rul.long}
%
\begin{Description}\relax
OTU transformation for longitudinal data. Computes average within-subject change 
(in presence for qualitative metrics, abundance for quantitative metrics) 
during one unit of time for each taxon.
\end{Description}
%
\begin{Usage}
\begin{verbatim}
tsf_long(otus, metadata)
\end{verbatim}
\end{Usage}
%
\begin{Arguments}
\begin{ldescription}
\item[\code{otus}] Matrix of OTU counts or proportions. Notes: (1) Will be transformed to 
proportions if it's not already; (2) Row names must be sample identifiers 
(matching metadata), and column names must be OTU identifiers (enforced if 
using UniFrac distances).

\item[\code{metadata}] Data frame with three columns: subject identifiers (n unique values, column name "subjID"), 
sample identifiers (must match row names of otu.tab, column name "sampID"), 
and time point or group identifier (if using longitudinal distances, this must be numeric or 
convertable to numeric).
\end{ldescription}
\end{Arguments}
%
\begin{Value}
List with the following elements. Both data matrices have subject identifiers 
as row names and OTU identifiers as column names.  
\begin{ldescription}
\item[\code{dat.binary}] n x p matrix of data after longitudinal, binary/qualitative transformation
\item[\code{dat.quant}] n x p matrix of data after longitudinal, quantitative transformation
\item[\code{avg.prop}] n x p matrix with overall average proportion of each taxon
\end{ldescription}
\end{Value}
\inputencoding{utf8}
\HeaderA{tsf\_paired}{tsf\_paired}{tsf.Rul.paired}
%
\begin{Description}\relax
OTU transformation for paired data. Computes within-subject change (in presence 
for qualitative metrics and abundance for quantitative metrics) between time 
points for each taxon.
\end{Description}
%
\begin{Usage}
\begin{verbatim}
tsf_paired(otus, metadata)
\end{verbatim}
\end{Usage}
%
\begin{Arguments}
\begin{ldescription}
\item[\code{otus}] Matrix of OTU counts or proportions. Notes: (1) Will be transformed to 
proportions if it's not already; (2) Row names must be sample identifiers 
(matching metadata), and column names must be OTU identifiers (enforced if 
using UniFrac distances).

\item[\code{metadata}] Data frame with three columns: subject identifiers (n unique values, column name "subjID"), 
sample identifiers (must match row names of otu.tab, column name "sampID"), 
and time point or group identifier (must have two unique values for paired transformation).
\end{ldescription}
\end{Arguments}
%
\begin{Value}
List with the following elements. Both data matrices have subject identifiers 
as row names and OTU identifiers as column names.  
\begin{ldescription}
\item[\code{dat.binary}] n x p matrix of data after paired, binary/qualitative transformation
\item[\code{dat.quant}] n x p matrix of data after paired, quantitative transformation
\item[\code{avg.prop}] n x p matrix with overall average proportion of each taxon
\end{ldescription}
\end{Value}
\inputencoding{utf8}
\HeaderA{unbal.long.meta}{Simulated metadata for balanced longitudinal study design.}{unbal.long.meta}
\keyword{datasets}{unbal.long.meta}
%
\begin{Description}\relax
Simulation code is included in the package vignette. 
Corresponding OTU matrix is stored in `unbal.long.otus`.
\end{Description}
%
\begin{Usage}
\begin{verbatim}
data(unbal.long.meta)
\end{verbatim}
\end{Usage}
%
\begin{Format}
A data frame with 14 rows and 3 columns. 
\begin{description}

\item[subjID] Subject identifiers
\item[sampID] Sample identifiers, matches row names of OTU count matrix
\item[time] Time indicator

\end{description}
\end{Format}
\inputencoding{utf8}
\HeaderA{unbal.long.otus}{Simulated OTU data for unbalanced longitudinal study design.}{unbal.long.otus}
\keyword{datasets}{unbal.long.otus}
%
\begin{Description}\relax
Simulation code is included in the package vignette. 
Corresponding metadata is stored in `unbal.long.meta`.
\end{Description}
%
\begin{Usage}
\begin{verbatim}
data(unbal.long.otus)
\end{verbatim}
\end{Usage}
%
\begin{Format}
A matrix with 14 rows and 10 columns. Rows are samples, columns are OTUs.
\end{Format}
\printindex{}
\end{document}
